














Note that we almost did not use smoothness of manifold $M$.fgh
In fact the Plaut's theorem makes possible to extend the proof above to complete length spaces;
that is, we get the following corollary from the proof.

\begin{thm}{Corollary}\label{cor:4(1)=>n(1)}
If a complete length space $X$ satisfies $4(1)$-tree comparison, then it satisfies any bipolar tree comparisons.
\end{thm}

For the cost-convex manifolds we have somewhat weaker statement.

\begin{thm}{Proposition}\label{prop:convexity-MTW}
A complete Riemannian manifold $M$ is cost-convex if and only if it is CTIL and
for any $p,q\in M$ the function $h\: \TIL_p\to \RR$ defined by
\[h(v)=\tfrac12\cdot|v|^2-\tfrac12\cdot\dist_q^2\circ\exp_p(v)\] 
has convex sub level sets.
\end{thm}

\parit{Proof; ``only if'' part.}
If $M$ is cost-convex, then Proposition~\ref{prop:CTIL} it is CTIL.

Fix $p\in M$ and $u,v\in \TIL_p$
and set $x=\exp_pu$ and $y=\exp_pv$.

Consider the cost-convex function 
\[f(s)=\min\{\cost(x,p)-\cost(x,s),\cost(y,p)-\cost(y,s)\}.\]

Note that the subgradient of $\ushort\nabla_pf$ is the line segment $[u\,v]$.
From the definition of cost-convex manifold (\ref{def:cost-convex}) it follows that if $u$ and $v$ lie in sub level set of $h$, then so is the segment $[u\,v]$.

\parit{``If'' part.} Follows from the propositions~\ref{prop:CTIL} and \ref{prop:CTIL}.
\qeds











These two conditions appear in the study of continuity of optimal transport between regular measures with positive continuous density functions.
The continuity implies both MTW and CTIL conditions and slightly stronger version of these two conditions imply the continuity;
see \cite{FRV-Nec+Suf}, \cite{MTW+CTIL} and the references there in.

%MORE???

Note that bipolar comparison provides a uniform way to treat combined CTIL+MTW condition;
this partially answers the question of Cédric Villani in ???.

Note that 4(1)-tree comparison (as well as any $T$-tree comparison) is stable under passing to the limit.
Namely, if a sequence of metric spaces $X_n$ satisfies 4(1)-comparison and converges in the sense of Gromov--Hausdorff to the space, then its limit $X_\infty$ also satisfies 4(1)-comparison.

We expect that for Riemannian manifolds, the 4(1)-tree comparison is equivalent to TCP and to MTW+CTIL (see question ???).
If this is indeed the case, then from above it will follow that the properties TCP and MTW+CTIL are stable as well.





















%%%%%%%%%%%%%%%%%%

Let $M$ be a Riemannian manifold.
The \emph{tangent injectivity locus} at the point $p\in M$ (briefly $\TIL_p$) is defined as the maximal open subset in the tangent space $\T_p$ such that for any $v\in\TIL_p$ the geodesic path $\gamma(t)=\exp_p(v\cdot t)$, $t\in [0,1]$ is a minimizing.
If the tangent injectivity locus at any point $p\in M$ is convex we say that $M$ satisfies \emph{convexity of  tangent injectivity locus} or briefly $M$ is CTIL.

 introduced a global differential geometric condition which is now called MTW, see .
The conditions CTIL and MTW are necessary for the regularity of optimal transport on Riemannian manifold $M$.
Moreover, a slightly stronger version of these conditions gives the converse.












Define the cost function
\[\cost(x,y)=\tfrac12\cdot|x-y|_M^2\]
on the Riemannian manifold $M$.
The functions which can be presented as exact upper bound for a nonempty set of functions of the type 
\[x\mapsto a- \cost(x,b)\]
for a $a\in \RR$ and $b\in M$ will be called cost-convex;
it takes values in $(-\infty,\infty]$.

Let $f\:M\to (-\infty,\infty]$ be a function which takes a finite value at $p\in M$.
Consider a subgradient $\ushort\nabla_pf\subset \T_p$; it is a possibly empty convex subset; see next section for precise definition.

The manifold $M$ is called \emph{cost-convex} if for any point $p\in M$ and costconvex function $f$ which takes finite value at $p$ for any $w\in\ushort\nabla_pf$ the geodesic path $t\mapsto \exp_p(t\cdot w)$ for $t\in [0,1]$ is minimizing
and 
\[\cost(q,p)-\cost(q,x)\ge f(x)-f(p)\]
holds for $q=\exp_pw$ and any $x\in M$.

It turns out that any cost-convex manifold is nonnegatively curved.

\parbf{4(1)-tree comparison.}
In Section~\ref{7-dipole}, we will show that the comparison for the tree 4(1) (see the diagram below) implies
the curvature condition introduced by Ma--Trudinger--Wang (briefly MTW condition) and convexity of tangent injectivity loci (briefly CTIL condition).














Let us denote by $\RR^\infty$ be the product of countably many real lines equipped with the product topology.

\begin{thm}{Lemma}\label{lem:tikhonov}
Let $\Gamma\acts \RR^\infty$ be a linear and continuous action of finitely generated group.
Assume there is a nonempty convex compact $\Gamma$-invariant set $\mathfrak{C}\subset \RR^\infty$.
Then the action has a fixed point in $\mathfrak{C}$.
\end{thm}

The following proof admits a straightforward generalization to the actions of finitely generated groups on locally convex spaces.
Possibly the finitely generated assumption can be removed.

\parit{Proof.}
Fix a set of generators $S=\{\gamma_1,\dots,\gamma_n\}$ in $\Gamma$.
Consider the average map
\[\phi(\mathfrak{r})=\tfrac1{n}\cdot(\gamma_1\cdot \mathfrak{r}+\dots+\gamma_n\cdot \mathfrak{r})\]
for $\mathfrak{r}\in\RR^\infty$.
Note that $\phi$ is $\Gamma$ invariant;
that is $\gamma\cdot \phi(\mathfrak{r})=\phi(\gamma\cdot\mathfrak{r})$ for any $\mathfrak{r}\in\RR^\infty$ and $\gamma\in \Gamma$.

Since $\mathfrak{C}$ is convex and $\Gamma$-invariant, $\phi$ maps $\mathfrak{C}$ in itself.
The subset of $\mathfrak{C}$ of all fixed points of $\phi$ is a convex closed $\Gamma$-invariant.
Moreover, by Tikhinov's this set of fixed points is not empty.

Without loss of generality, we can assume that $\mathfrak{C}$ is minimal (with respect to inclusion) set satisfying the assumption of the lemma.
In this case, $\mathfrak{C}$ contains only fixed vectors;
that is, $\phi(\mathfrak{r})=\mathfrak{r}$ for any $\mathfrak{r}\in \mathfrak{C}$.

Denote by $x_i(\mathfrak{r})$ the $i$-th coordinate of $\mathfrak{r}$.
Fix $i$ and choose $\mathfrak{r}\in \mathfrak{C}$ so that $x_i(\mathfrak{r})$ takes the maximal value.
Since $\phi(\mathfrak{r})=\mathfrak{r}$, we get 
\[x_i(\mathfrak{r})=x_i(\gamma\cdot \mathfrak{r})\]
for all $\gamma\in\Gamma$.

Since $\mathfrak{C}$ is minimal, $x_i$ is constant on $\mathfrak{C}$.
Since $i$ is arbitrary, the statement follows.
\qeds



The fundamental group $\Gamma=\pi_1K_n$ acts on $\~ K_n$ by deck transformations.
Let us show that the map $f$ can be chousen so that the action of $\Gamma$ extends to an isometric action of $\HH$.
That is, there is an isometric actio $\Gamma\acts \HH$, such that $f(\gamma\cdot \~v)=\gamma\cdot f(\~v)$ for any vertex $\~v\in\~V$. %???right-or-left

Take a copy of the real line $\RR$ for each pair of vertexes $\~v,\~w$ in $\~V$ and
consider the product space $\RR^\infty$ of all these lines. 
The group $\Gamma$ naturally acts on $\RR^\infty$ by permuting coordinates.

Denote by $\mathfrak{r}_f$ the point in $\RR^\infty$ with the coordinates $x_{\~v,\~w}=|f(\~v)-f(\~w)|_\HH^2$ for each pair $(\~v,\~w)$ of vertexes in $\~V$.
Note that for any $\gamma\in\Gamma$, we have
\[\gamma\cdot\mathfrak{r}_f=\mathfrak{r}_{\gamma\cdot f},\]
where $(\gamma\cdot f)(\~v):= f(\gamma\cdot \~v)$.
 
Denote by $\mathfrak{C}$ the set of all vectors $\mathfrak{r}_f\in\RR^\infty$ for the maps $f$ which satisfy the condition $({*})$.

Note that $\mathfrak{C}$ is a compact subset in $\RR^\infty$.

Indeed, evidently $\mathfrak{C}$ is closed.
Further, for any pair of vertexes $\~v,\~w\in\~V$, there is a path $\~v=\~v_0,\dots,\~v_k=\~w$ so that each pair $(\~v_{i-1},\~v_i)$ are adjacent.
Set 
\[s_{\~v,\~w}=|a_{v_0}-a_{v_1}|_X+\dots+|a_{v_{k-1}}-a_{v_k}|_X.\]
By triangle inequality 
\[|f(\~v)-f(\~w)|_\HH\le s_{v,w}\]
It follows that $\mathfrak{C}$ lies in the product of the intervals $[0,s_{\~v,\~w}^2]$
for all pairs $(\~v,\~w)$ of vertexes in $\~V$.
By Tikhonov's theorem, the product of these intervals is compact;
hence so is~$\mathfrak{C}$.

Note that $\mathfrak{C}$ is a convex subset of $\RR^\infty$.

Indeed, assume $f,h\:\~V\to H$ be two maps satisfying $({*})$.
Fix $\alpha\in[0,\tfrac\pi2]$ and 
consider the map $g\:K_n\to \HH=\HH\times \HH$ defined by
\[g(v)=(\cos\alpha\cdot f(v),\sin\alpha\cdot f(v))\]
Note that $g$ satisfies the condition $({*})$ and 
\[\mathfrak{r}_g=(\cos\alpha)^2\cdot\mathfrak{r}_f+(\sin\alpha)^2\cdot\mathfrak{r}_h;\]
that is, $\mathfrak{r}_g$ is the convex combination of  $\mathfrak{r}_f$ and $\mathfrak{r}_h$ with the weights  $(\cos\alpha)^2$ and $(\sin\alpha)^2$.
Hence the convexity of $\mathfrak{C}$ follows.

By Lemma~\ref{lem:tikhonov}, the action $\Gamma\acts\RR^\infty$ has a fixed point in $\mathfrak{C}$.
The corresponding map $f\:\~V\to \HH$ has the needed property.

Indeed, let $\HH_1$ be the minimal affine subspace of $\HH$ containing $f(\~V)$.
Then the map $\iota_\gamma\:f(v)\mapsto f(\gamma\cdot v)$ is distance preserving.
It follows that $\iota_\gamma$ admits a unique extension to an isometry $\bar\iota_\gamma\:\HH_1\to\HH_1$.
The map $\gamma\mapsto \bar \iota_\gamma$ describes an isometric group action of $\Gamma$ on $\HH_1$.

It remains to extend the obtained action on whole $\HH$;
for example as the diagonal action on $\HH=\HH_1\times\HH_2$,
where $\HH_2$ be the orthogonal complement of $\HH_1$ in $\HH$.