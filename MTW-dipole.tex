\documentclass[a4paper,12pt]{article}
%\usepackage{kusochek}\input{point4.tex}
\usepackage{amsmath}
\usepackage{amsthm}
\usepackage{amsfonts, amssymb}
\usepackage{eucal}
\usepackage{xcolor}
%\usepackage{kubik}
\def\ann #1#2#3{ \can({#1}^{#2}_{#3})} 

\newtheorem{theorem}{Theorem}
\newtheorem*{theorem*}{Theorem}

\newtheorem{lemma}[theorem]{Lemma}
\newtheorem{proposition}[theorem]{Proposition}
\newtheorem{conjecture}[theorem]{Conjecture}
\newtheorem{corollary}[theorem]{Corollary}
\theoremstyle{definition}
\newtheorem{remark}[theorem]{Remark}
\theoremstyle{definition}
\newtheorem{example}[theorem]{Example}
%\newtheorem{remark}[theorem]{Remark}
\theoremstyle{definition}
\newtheorem{definition}[theorem]{Definition}


\begin{document}
%%%%%%%%%%%%%% Real numbers
\def\R{{\mathbb R}}
%%%%%%%%%%%%%% Expectation
\def\E{{\mathbb E\,}}
%%%%%%%%%%%%%% Probability
\def\p{{\mathbb P}}
\def\P{{\mathbb P}}
%%%%%%%%%%%%%% Integers
\def\z{{\mathbb Z}}
\def\Z{{\mathbb Z}}
%%%%%%%%%%%%%% Natural numbers
\def\n{{\mathbb N}}

\def\SS{{\mathbb S}}
\def\DD{{\mathcal D}}


\def\N{{\mathbb N}}

\def\c{{\mathbb C}}
\def\C{{\mathbb C}}
\def\wt{\widetilde}
\def \wh{\widehat}

\def\ann #1#2#3{ \can({#1}^{#2}_{#3})} 

\def\ns{{\mathbb N}^*}

\def\ed#1{ {\mathbf 1}_{ \{#1  \}}}             % indicator
\def\e2d#1#2{ {\mathbf 1}_{ \{#1,#2  \}}}             % indicator
\def\AA{{\mathcal A}}
\def\BB{{\mathcal B}}
\def\FF{{\mathcal Q}}
\def\GG{{\mathcal G}}
\def\NN{{\mathcal N}}
\def\PP{{\mathcal P}}
\def\PPP{{\mathscr{P}}}
\def\TT{{\mathcal T}}
\def\w{{\mathcal W}}
\def\M{{\mathcal M}}
%\def\S{{\mathcal S}}
\def\Chi{{\mathcal X}}
\def\Tau{{\mathcal T}}
\def\Nu{{\mathcal V}}
\def\qq{{\widetilde{q}}}
\def \EE{{\mathcal{E}}}
\def\<{\langle}
\def\>{\rangle}
\def \ee{{\epsilon}}
\def \dd{{\delta}}
\def \ss{{\sigma}}
\def \aa{{\alpha}}
\def \ll {{\lambda}}
\def \oa {\overrightarrow}

\def \an {\measuredangle}
\def \can {\wt \an}
\def\ann #1#2#3{ \can({#1}^{#2}_{#3})}             % indicator
 
\begin{definition} We say that a metric space $(M, d )$ satisfies
 $(k,l)$-dipole property
if for any 
$A_0, B_0\in M$ (poles) and any
$A_1,\dots, A_k, B_1,\dots, B_l\in M$ there is a map
$f:\{A_0, A_1,\dots, A_k, B_0, B_1,\dots, B_l\}\to \R^{k+l+1}$ such that
$$|f(A_0)f(B_0)|=d(A_0, B_0),|f(A_0)f(A_i)|=d(A_0, A_i), |f(B_0)f(B_j)|=d(B_0, B_j)$$
and
$d(A_i, B_j)\le |f(A_i)f(B_j)|$ for
$0\le i\le k, 0\le j \le l$.
\end{definition}

\begin{remark}
The $(k,0)$-dipole property 
coincide with $(1+n)$-point comparison (for n=k+1)
from
["Alexandrov meets Kirszbraun"
S. Alexander, V. Kapovitch, A. Petrunin].
$(1+n)$-point comparison  is
satisfied for any $CBB(0)$ space
and $(1+3)$-point comparison
is characterizing for $CBB(0)$ among complete spaces
with intrinsic metric.



\end{remark}



\begin{proposition}
Let
M
be a Riemannian manifold satisfying
CTIL. Then if $M$ satisfies 
$(7,1)$-dipole property then $M$ satisfies MTW. 

\end{proposition}

{\it proof}

We want to verify conditions of reformulation of MTW as in
Proposition 2.6 from here
 http://cedricvillani.org/wp-content/uploads/2012/08/038.4Curv.pdf
Let $\bar x, y, y_0, y_1, x$ be as in
Proposition 2.6.

We denote
$v:=\overrightarrow{\bar x y}$,  $v_0:=\overrightarrow{\bar x y_0}$,
$v_1:=\overrightarrow{\bar x y_1}$.

Let $\varepsilon>0$ and set

$$A_0=\bar{x}, B_0=y$$
$$A_1=y_0, A_2=y,  A_3=y_1$$ 
$$ A_4^{\varepsilon}=\exp_{A_0}(-\varepsilon v),
A_5^{\varepsilon}=\exp_{A_0}( -\varepsilon v_0),
A_6^{\varepsilon}=\exp_{A_0}(-\varepsilon v_1),
A_7^{\varepsilon}=\exp_{A_0}(\varepsilon v)$$
$$B_1=x$$

Now let $f_{\varepsilon}$ be 
 the map provided by Definition 1:
$$f_{\varepsilon}:\{A_0, A_1,A_2, A_3, A_4^{\varepsilon},
A_5^{\varepsilon}, A_6^{\varepsilon}, A_7^{\varepsilon}, B_0, B_1\}\to \R^9.$$

Let note, that if
 $\varepsilon$ is small the map
$f_\varepsilon$   is close to $\exp_{A_0}^{-1}$ on the set $\{A_0, A_1,A_2, A_3, A_4^{\varepsilon},
A_5^{\varepsilon}, A_6^{\varepsilon}, A_7^{\varepsilon}, B_0\}$
(we assume here that $T_{A_0}M\subset\R^9$).
Indeed, triples of points that lie on one minimizing (we  can assume them to be
 minimizing,
otherwise we could move $y$, $y_0$, $y_1$ slightly toward $\bar x$ ) geodesics, such
as $$A_1, A_0, A_5^\varepsilon, \qquad
A_2, A_0, A_4^\varepsilon,  \qquad A_3, A_0, A_6^\varepsilon$$
go to the points on one segment. 
On the other hand angles between $A_0 A_4^{\varepsilon}$,
$A_0A_5^{\varepsilon}$,
$A_0A_6^{\varepsilon}$,
$A_0A_7^{\varepsilon}$ are almost preserved.

Now inequality
(2.4)
follows by passing to the limit, because it is fulfilled in $\R^9$
and because of relations in Definition 1.

\qed\par\medskip

PS.  It seems that
in the formulation of Proposition 2.6 geodesics $\gamma_0$ and $\gamma_1$ 
should be minimizing (?).

\section{(2,2)-dipole}

\begin{theorem}
The $(2,2)$-dipole property holds for
any $CBB(0)$ space.


\end{theorem}

In this section $(M,d)$ denotes $CBB(0)$ space.

\subsection{Comparison in Alexandrov space }

In our proof we use two comparison facts from Alexandrov geometry.
The next lemma will be applied for $n=3,4$.

\begin{lemma}[Simplex comparison]\label{comparison}
Let $(M, d)$ be a $CBB(0)$ space.
Let
$C_1^*,\dots , C_n^*\subset \R^{n-1}$ be in general position
a point
$C^*\in int (conv \{C_1^*,\dots , C_n^*\})$.  Let
$C, C_1,\dots, C_n\in M$  be such that
$d(C, C_i)\le |C^*C_i^*|$ for $1\le i\le n$ and $d(C_i, C_j)\ge|C_i^*C_j^*|$
for $(i,j)\in \{1,\dots, n\}\times\{1,\dots,n\}\setminus \{(1,n)\}$.
Then
$d(C_1,C_n)\le |C_1^*C_n^*|$.

\end{lemma}

\it{proof}
Let
$\lambda_1^*,\dots,\lambda_n^*>0$, $\sum_{i=1}^n\lambda_i^*=1$
be  barycentric coordinates
of $C^*$ w.r.t. $C_1^*,\dots, C_n^*$, 
then
$$2\sum_{i=1}^n \lambda_i^* |C^*C_i^*|^2=\sum_{i,j=1}^n \lambda_i^*\lambda_j^*|C_i^* C_j^*|^2. $$

It is known that for any
$\lambda_1,\dots,\lambda_n>0$, $\sum_{i=1}^n\lambda_i=1$ and
all $X_0, X_1,\dots, X_n\in M$ the following inequality holds
$$2\sum_{i=1}^n \lambda_i d(X_0,X_n)^2\ge \sum_{i,j=1}^n \lambda_i\lambda_jd(X_i, X_j)^2, $$
in particular
$$2\sum_{i=1}^n \lambda_i^* d(C,  C_i)^2
\ge \sum_{i,j=1}^n \lambda_i^*\lambda_j^*d(C_i, C_j)^2. $$
Comparing this with the above equality in $\R^n$ and
taking in account conditions of the lemma gives the required inequality.

          
    \qed\par\medskip
         
\begin{lemma}[Quaigeodesic comparison]\label{QG}
Let $A\in M$ and
$\gamma:[0,\infty)\to M$
be a unit speed quasigeodesic, let
$A^*\in\R^n$ and
$\gamma^*:[0,\infty)\to \R^n$
be a unit speed ray such that
$d(A,\gamma(0))=|A^*\gamma^*(0)|$
and $\angle(\uparrow_{\gamma(0)}^ A, \gamma'_+(0))=
\angle(\uparrow_{\gamma^*(0)}^{A^*}, (\gamma^*)_+'(0)) $
then $d(A,\gamma(t))\le |A^*,\gamma^*(t)|$.

\end{lemma}

This follows from definition of quasigeodesisic (\ref{}).

\subsection{Comparison construction}
 Let point $A_0, A_1, A_2, B_0, B_1, B_2\in M$.
 We prove the $(2,2)$-dipole comparison for the case when
  $A_0, B_0$ are regular points (that is tangent spaces are Euclidean). 
 The general case (if the dimension is finite)  can be obtained by approximation of these points
with regular, and for infinite dimension there could be applied more complicated trick,
described in AKP.  
 
 Let us describe comparison construction for these points in
   % for points $A_0, A_1, A_2, B_0, B_1, B_2$ in
    $R^4$.
    Comparison points $A_0', B_0'\in\R^4 $ are fixed in such a way that
    $|A_0'B_0'|=d(A_0, B_0)$
and    $O$ is a midpoint $[A_0'B_0']$. 
Comparison point for each of the remaining points 
depends on additional parameter vector. 
Let $S^2$ be the
    two dimensional unit sphere with center $O$ and 
    in the plane orthogonal to $[A_0'B_0']$. (We identify further vectors
    and their endpoint on the sphere).
 For every  vector $e\in S^2$ and every point $X\in\{A_1, A_2, B_1, B_2\}$
  we  define  
   comparison point $X(e)\in\R^4$
 by the following conditions (we write for $X=A_1, A_2$, for $X=B_1,B_2$ conditions are
 defined in the same way):
   
    1) $|A_i(e)A_0'|=d(A_i, A_0)$
    
    2)$\angle (A_i(e)A_0'B_0')=\angle(A_i A_0 B_0)$
    
    3) segment $[A_0'A_i(e)]$ lies in a half-plane,    with a boundary line $(A_0'B_0')$
    and direction $e$.
    
Let note that for different
$X, Y \in\{A_1, A_2, B_1, B_2\}$ and $e_X, e_Y\in S^2$
the distance 
$|X(e_X) Y(e_Y)|$ depends only on the angle $|e_Xe_Y|_{S^2}$
and we can then correctly define auxiliary  function
$f_{XY}:[0,2\pi]\to[0,\infty)$, which maps
$|e_Xe_Y|_{S^2}\mapsto |X(e_X) Y(e_Y)|$.
Obviously $f_{XY}$ is nondecreasing function,
and we want to show next that 
$f_{XY}(\pi)\ge d(X,Y)$

Let us define the curve $\gamma:\R\to M$ which we use
further for comparison several times.
We denote $t_0=d(A_0,B_0)$ and let
$\gamma_{[0, t_0]}$ be a minimizing unit speed geodesic
 between $A_0$ and $B_0$.
 We extend $\gamma$ on the rays
 $(-\infty, 0]$ and $[t_0,\infty)$ as quasigeodesics
 with initial vectors
 opposite to the right and left derivatives correspondingly of $\gamma_{[0, t_0]}$ in the end points.
  We also denote by 
 $\gamma_*:\R\to \R^4$ a unit speed line with
  $\gamma_*(0)=A_0'$ and $\gamma_*(t_0)=B_0'$. 


Let us take some opposite vectors $e_X=- e_Y\in S^2$ (this implies in particular
that $f_{XY}(\pi)=|X(e_X) Y(e_Y)|$).
Then $[X(e_X), Y(e_Y)]$ intersect line $\gamma_*(\R)$ in a point say
$C_*=\gamma_*(t_*)$.  Denote  $C=\gamma(t_*)$.
By comparison $d(C,X)\le |C_*X(e_X)|$ and $d(C,Y)\le |C_*Y(e_Y)|$.
Then by triangle inequality $d(X,Y)\le |X(e_X) Y(e_Y)|=f_{XY}(\pi)$.


This makes possible to define an angle $\alpha_{XY}$ (which plays a key role
in our next comparisons) as follows. We set
  $\alpha_{XY}:= f^{-1}(d(X,Y))$  if $d(X,Y)\ge f_{XY}(0)$
  and   $\alpha_{XY}:=0$
  otherwise. 
  This function has the following property.
  
  
  
   \begin{lemma}\label{<2pi}
  
 
   $\alpha_{XY}+\alpha_{YZ}+\alpha_{ZX}\le2\pi$
    
     \end{lemma}
     
     {\it proof}
     We can choose $e_X, e_Y, e_Z\in S^2$ so that
     
    1)  they are linearly dependant
    
    2) $|e_Xe_Y|_{S^2}=\alpha_{XY}$
     and $|e_Xe_Z|_{S^2}=\alpha_{XZ}$.
     
     If $\alpha_{XY}   + \alpha_{XZ}\le \pi$ the conclusion is trivial,
     suppose the opposite. Then convex hull
     of $\{X(e_X), Y(e_Y), Z(e_Z)\}$ intersects line $\gamma_*$
     in a point say
$C_*=\gamma_*(t_*)$.  Denote  $C=\gamma(t_*)$.
We have that
$|X(e_X), Y(e_Y)|=d(X,Y)$
and $|X(e_X)Z(e_Z)=d(X,Z)$.
By quasigeodesic comparison (Lemma~\ref{comparison}) we have $d(C,X)\le |C_*X(e_X)|$ and $d(C,Y)\le |C_*Y(e_Y)|$. Then we can apply
simplex comparison (Lemma~\ref{comparison}) and obtain
that 
$|Y(e_Y)Z(e_Z)|\ge d(Y,Z)$.
Hence $|e_Ye_Z|_{S^2}\ge \alpha_{XY}$,
this implies
 the statement
of the lemma.
          
    \qed\par\medskip
         
   
    \subsection{The proof }
    
       Let us fix $X,Y\in\{A_1, A_2, B_1, B_2\}$ such that $\alpha_{XY}$ is maximal,
    and denote $Z,W$ the remaining two points.
    
   
    Suppose that
    triangle inequality hold for $\alpha_{XY}, \alpha_{YZ}, \alpha_{XZ}$ and for
   $\alpha_{XY}, \alpha_{YW}, \alpha_{XW}$.    
   Then (in view of Lemma~\ref{<2pi}) we can find  points  $e_X, e_Y,e_Z,e_W\in S^2$ be such that
   $$|e_Xe_Y|_{S^2}=\alpha_{XY}, |e_Xe_Z|_{S^2}=\alpha_{XZ},   
   |e_Ze_Y|_{S^2}=\alpha_{ZY}, |e_Xe_W|_{S^2}=\alpha_{XW}, |e_We_Y|_{S^2}=\alpha_{WY}$$   
   and $e_Z$ and $e_W$ (as points in $S^2$) lie in different semispheres
   with boundary equator containing $[e_X,e_Y]$.
   
   Our proof splits in two cases:
    
    1) the above mentioned triangle inequality hold and constructed points
     $e_X, e_Y,e_Z,e_W\in S^2$ doesn't lie in any semisphere {\bf(*)}
     
     2) (*) doesn't hold
     
     
     \subsubsection{proof for the case 1)}
     In this case we have that
     convex hull
     of $\{X(e_X), Y(e_Y), Z(e_Z), W(e_W)\}$ intersects line $\gamma_*$
     in a point say
$C_*=\gamma_*(t_*)$.      
  We have that
$$|X(e_X), Y(e_Y)|=d(X,Y),
|X(e_X)Z(e_Z)=d(X,Z)
|X(e_X), Y(e_W)|=d(X,W)$$
$$|X(e_Y)Z(e_Z)=d(Y,Z),
|X(e_Y), Y(e_W)|=d(Y,W)$$
By quasigeodesic comparison (Lemma~\ref{comparison}) we have $d(C,X)\le |C_*X(e_X)|$,
$d(C,Y)\le |C_*Y(e_Y)|$,
 $d(C,Z)\le |C_*Z(e_Z)|$ and
$d(C,W)\le |C_*W(e_W)|$.
Then we can apply
simplex comparison (Lemma~\ref{comparison}) and 
obtain
that 
$|Z(e_Z)W(e_W)|\ge d(Z,W)$.
Hence points
$A_0', B_0', X(e_X), Y(e_Y), Z(e_Z), W(e_W)$
satisfy required comparison inequalities.

       \qed\par\medskip     
     
     
     
      \subsubsection{proof for the case 2)}
      
      In this case we don't rely any more on Alexandrov space comparison.
      Our problem in this case can be reformulated as follows.
      We have $6$  numbers 
      $\alpha_{XY}, \alpha_{XZ}, \alpha_{XW}, \alpha_{YZ}, \alpha_{YW},
      \alpha_{ZW}\in[0,2\pi]$, which can be imaging as distances between 
      $4$ points without triangle inequality,  such that perimeter 
        of all triangles is bounded by $2\pi$ (Lemma~\ref{<2pi})
           and (*) doesn't hold.
           Then it is sufficient to find a non-contracting map from
           this $4$-point  space to $S^2$.
           Indeed, suppose
           points
    $e_{X}, e_{Y}, e_{Z}, e_{W}\in S^2$
    be the image of such a map,    
         then points
$A_0', B_0', X(e_X), Y(e_Y), Z(e_Z), W(e_W)$
satisfy required comparison inequalities.


Our proof splits further in cases.
We can assume w.l.o.g. that 
$\alpha_{XZ}+\alpha_{XW}\ge\alpha_{YZ}+\alpha_{YW}$.

(1) triangle inequality hold for $\alpha_{XY}, \alpha_{YZ}, \alpha_{XZ}$ and for
   $\alpha_{XY}, \alpha_{YW}, \alpha_{XW}$ and
   
   (1.a) $\alpha_{XZ}+\alpha_{XW}\ge\pi$   
     
     (1.b)      $\alpha_{XZ}+\alpha_{XW}<\pi$      
     
     (2) exactly one of the above triangle inequality hold,
     we can assume it holds for the triple $\alpha_{XY}, \alpha_{YZ}, \alpha_{XZ}$  and
     
     
   (2.a) $\alpha_{XZ}+\alpha_{XW}\ge\pi$   
     
     (2.b)      $\alpha_{XZ}+\alpha_{XW}<\pi$           
     
     (3) both  above triangle inequalities fail and    
     
   (3.a) $\alpha_{XZ}+\alpha_{XW}\ge\pi$   
     
     (3.b)      $\alpha_{XZ}+\alpha_{XW}<\pi$ 
            
\noindent
     {\it Proof of (1.a)}
     Let $e_X', e_Y',e_Z', e_W'\in S^2$ be such that
   $$|e_X'e_Y'|_{S^2}=\alpha_{XY}, |e_X'e_Z'|_{S^2}=\alpha_{XZ},   
   |e_Z'e_Y'|_{S^2}=\alpha_{ZY}, |e_X'e_W'|_{S^2}=\alpha_{XW}, |e_W'e_Y'|_{S^2}=\alpha_{WY}$$   
   and $e_Z'$ and $e_W'$ (as points in $S^2$) lie in different semispheres
   with boundary equator containing $[e_X',e_Y']$.
     Then $e_X', e_Y',e_Z', e_W'$ lie in one semishere and condition
     $\alpha_{XZ}+\alpha_{XW}\ge\alpha_{YZ}+\alpha_{YW}$     
     imply that $\angle_{S^2}(e_Y'e_X'e_Z')+\angle_{S^2}(e_Y'e_X'e_W')\le \pi$.
     We then set $e_X=e_X', e_Y=e_Y', e_Z=e_Z'$ and take $e_W$
      on the prolongation of the geodesic (in $S^2$) $[e_Ze_X]$ after $e_X$, such that 
     $|e_We_X|_{S^2}=\alpha_{XW}$.
     Then $|e_Ye_W|_{S^2}\ge\alpha_{YW}$
     because by construction $\angle_{S^2}(e_Ye_Xe_W)\ge\angle_{S^2}(e_Y'e_X'e_W')$
     and $|e_Xe_W|_{S^2}\ge\alpha_{YW}$ because of Lemma~\ref{<2pi}.     
        So the required inequalities hold for distances between
        $e_X, e_Y, e_Z, e_W$. 
        
        \noindent         {\it Proof of (1.b)}       
        In this case we can take $e_X, e_Y,e_Z\in S^2$       
        such that   $$|e_Xe_Y|_{S^2}=\alpha_{XY}, |e_Xe_Z|_{S^2}=\alpha_{XZ},   
   |e_Ze_Y|_{S^2}=\alpha_{ZY}$$   and $e_W=-e_Z$.     
        
                     
\noindent
     {\it Proof of (2.a)}In this case we can take $e_X, e_Y,e_Z\in S^2$       
        such that   $$|e_Xe_Y|_{S^2}=\alpha_{XY}, |e_Xe_Z|_{S^2}=\alpha_{XZ},   
   |e_Ze_Y|_{S^2}=\alpha_{ZY}$$   and
$e_W$
      on the prolongation of the geodesic (in $S^2$) $[e_Ze_X]$ after $e_X$, such that 
     $|e_We_X|_{S^2}=\alpha_{XW}$.
     
     
\noindent         {\it Proof of (2.b)}    We can take the same construction as in (1.b).


\noindent         {\it Proof of (3.a)}    
We take  $e_X, e_Y\in S^2$       
        such that   $|e_Xe_Y|_{S^2}=\alpha_{XY}$ and
        $e_Z, e_W$ on any equator containing $e_X$
        in opposite directions from $e_X$ and such that 
 $|e_Xe_Z|_{S^2}=\alpha_{XZ},   
   |e_Xe_W|_{S^2}=\alpha_{XW}$.


\noindent         {\it Proof of (3.b)} 
To simplify the text we want to make our notation for this case
symmetric w.r.t. $X$ and $Y$, so we 'forget' condition
$\alpha_{XZ}+\alpha_{XW}\ge\alpha_{YZ}+\alpha_{YW}$
and (3.b) is formulated then
$\max\{\alpha_{XZ}+\alpha_{XW}, \alpha_{YZ}+\alpha_{YW}\}<\pi$.

Let us fix point $e\in S^2$
and take $e_X, e_Y$ such that $|e_Xe_Y|_{S^2}=\alpha_{XY}$ and $e$ is a midpoint between them.
We can assume that $$\alpha_{XZ}=\max\{\alpha_{XZ}, \alpha_{XW}, 
\alpha_{YZ}, \alpha_{YW}\}.$$

 Then we take a point
$e_Z$ on the equator w.r.t. pole $e$ and such that $|e_Ze_X|_{S^2}=\alpha_{XZ}$
if such a point exists (denote this case (3.b.1)) and take arbitrary point on this equator if doesn't
 (denote this case as (3.b.2); one can see from the picture that in this
case $|e_Ze_X|_{S^2}>\alpha_{XZ}$ because $\alpha_{XY}\ge\alpha_{XZ}$).
We take $e_W=-e_Z$.

Now in the case (3.b.1)
we have $|e_Ze_Y|_{S^2}\ge\alpha_{ZY}$, because $\alpha_{XY}>\alpha_{XZ}+\alpha_{YZ}$.
We also have $|e_We_Y|\ge\alpha_{WY}$, because $\alpha_{WY}\le\alpha_{XZ}$.
And we have 
$|e_We_X|_{S^2}+|e_Xe_Z|_{S^2}=\pi$, while $\alpha_{XW}+\alpha_{XZ}<\pi$,
hence $|e_We_X|_{S^2}<\alpha_{XW}$.

In the case (3.b.2) all required inequalities follow since
 $\alpha_{XZ}=$
 
 \noindent
  $=\max\{\alpha_{XZ}, \alpha_{XW}, 
\alpha_{YZ}, \alpha_{YW}\}$.

        
   \bibliography{Sturm}
\bibliographystyle{plain}
  
\end{document}