\section{Pull-back convexity}\label{convexity}

In this section reformulate 4(1)-tree comparison using convexity of certain function on tangent space and use it to prove the second part of Theorem \ref{T=>CTIL:CTIL}.

\begin{thm}{Proposition}\label{prop:convexity}
If a Riemannian manifold $M$ satisfies $4(1)$-tree comparison then it is CTIL and the for any $p,q\in M$ the function $f\: \TIL_p\to \RR$ defined by
\[f(v)=\tfrac12\cdot\dist_q^2\circ\exp_p(v)-\tfrac12\cdot|v|^2\] 
is concave.

Converse also holds; moreover, if $f$ is convex for any $p,q\in M$ then $M$ satisfies all bipolar trees comparisons; that is the $m(n)$-tree comparison for any $m$ and $n$.
\end{thm}



Recall that for a continuous function $f$ defined on metric space we write 
$f''\le \lambda$ if for any unit-speed geodesic $\gamma$ the function
\[t\mapsto f\circ\gamma(t)-\lambda\cdot \tfrac{t^2}{2}\]
is a concave real-to-real function.

\parit{Proof.}
Assume $M$ satisfies 4(1)-tree comparison.
Since 4(1)-tree comparison implies 3-tree comparison,
$M$ has nonnegative sectional curvature.

Fix $u,v\in \TIL_p$ and $w\in [u,v]$.
Assume that $g\:[u,v]\:\RR$ is the function which is uniquely defined by conditions:
\[g''=1,\quad
g(u)=f(u)\quad
\text{and}\quad
g(w)=f(w).\]
It is sufficient to show that $g(v)\ge f(v)$.

Fix small $\eps>0$ and set
\begin{align*}
x&=\exp_p u, 
&y&=\exp_p v, 
&z&=\exp_pw,
\\
x'&=\exp_p(-\eps\cdot  u),
&y'&=\exp_p(-\eps\cdot  v)
\end{align*}
Apply the $p/xyx'y'(z/q)$ comparison and pass to the limit as $\eps\to 0$.
We obtain a configuration of points $\~p, \~x, \~y, \~z, \~q\in\HH$, satisfying corresponding comparisons and
such that
\begin{align*}
\measuredangle[\~p\,^{\~x}_{\~y}]&= \measuredangle[p\,^{x}_{y}],
&\measuredangle[\~p\,^{\~x}_{\~z}]&= \measuredangle[p\,^{x}_{z}],
&\measuredangle[\~p\,^{\~z}_{\~y}]&= \measuredangle[p\,^{z}_{y}].
\end{align*}
In particular,
from above and Toponogov comparison, we have
\begin{align*}
|\~x-\~y|_\HH&=|u-v|_{\T_p},
&|\~z-\~y|_\HH&=|w-v|_{\T_p},
&|\~x-\~z|_\HH&=|u-w|_{\T_p},
\\
|\~q-\~z|_\HH&=|q-z|_M,
&|\~q-\~x|_\HH&\ge|q-x|_M,
&|\~q-\~y|_\HH&\ge|q-y|_M.
\end{align*}

Let $q^*\in \HH$ be the point
such that 
\begin{align*}
|q^*-\~z|&=|q-z|, 
&|q^*-\~x|&=|q-x|
\end{align*}

Assume
$\varphi\:[u,v]_{\T_p}\to[\~x,\~y]_{\HH}$ is an isometry such that
$\varphi(u)=\~x$, $\varphi(v)=\~y$ and
$\varphi(w)=\~z$.
Then $g:=\dist^2_{q^*}\circ\varphi$, indeed, we have, that $g''=1$ and
$$g(u)=|q^*-\~x|^2=|q-x|^2=f(u),\quad
g(w)=|q^*-\~z|^2=|q-z|^2=f(w)$$
$$g(v)=|q^*-\~y|^2=|q-y|^2\ge f(u).$$
Hence the statement.

\parit{Converse.}
Fix points $p$ and $q$ in $M$;
set $\~q=\log_pq\in\T_p$.
Note that 
\[f\le \~f,\eqlbl{eq:f=<f}\] 
where 
\[\~f(v)=\tfrac12\cdot |v-\~q|_{\T_p}^2-\tfrac12\cdot |v|_{\T_p}^2.\]
Further note that the inequality \ref{eq:f=<f} is equivalent to the Toponogov comparison for all hinges $[p\,{}^x_q]$ in $M$.
It follows that $M$ has nonnegative sectional curvature. 

\medskip

Now fix a bipolar geodesic tree $[p/x_1\dots x_n(q/y_1\dots y_m)]$ in $M$.
Set 
\[\~p=0=\log_pp,\quad \~q=\log_pq,\quad\text{and}\quad \~x_i=\log_px_i\]
for each $i$. 

Consider the linear map $\psi\:\T_q\to \T_p$ such that for any smooth function $h$
\[\psi_1\:\nabla_{q}h\mapsto \nabla_{\~q}(h\circ\exp_p).\]
Since sectional curvature of $M$ is nonnegative, the restriction $\exp_p|_{\TIL_p}$ is short and therefore $\psi_1$ is short.

In particular there is a linear map $\psi_2\:\T_q\to\T_p$ such that, the map $\iota\:\T_q\zz\to \T_p\oplus\T_p$ defined by
\[\iota\:v\mapsto \psi_1(v)\oplus \psi_2(v)\]
is distance preserving.

Further set 
\[h_i=\tfrac12\cdot\dist_{y_i}^2
\quad
f_i=h_i\circ\exp_p|_{\TIL_p}
\quad
\~y_i=\~q-\zeta(\nabla_q h_i).
\]


By construction
\[|\~y_i-\~q|_{\T_p\oplus\T_p}=|y_i-q|_M.\]

At the point $\~q$ the restriction functions $\~f_i=\tfrac12\cdot\dist^2_{\~w}|_{\T_p}$ and the function $f_i$ have the same value and gradient.
Since $f_i''\le 1$ and $\~f_i''=1$, we get $\~f_i\ge f_i$. 
The latter implies
\[
|\~y_i-\~p|_{\T_p\oplus\T_p}\ge|y_i-p|_M
\quad
\text{and}
\quad
|\~y_i-\~x_j|_{\T_p\oplus\T_p}\ge|y_i-x_j|_M.\]
for any $i$ and $j$.

Since $\T_p\oplus\T_p$ admits an isometric embedding into the Hilbert space,
we get the needed configuration.
\qeds

The Plaut's theorem makes possible to extend the proof above to complete length spaces;
that is, we get the following corollary from the proof.

\begin{thm}{Corollary}\label{cor:4(1)=>n(1)}
If a complete length space $X$ satisfies $4(1)$-tree comparison then it satisfies any bipolar tree comparisons.
\end{thm}

For MTW manifolds we have somewhat weaker statement.

\begin{thm}{Proposition}\label{prop:convexity}
A CTIL Riemannian manifold $M$ is MTW if and only if 
for any $p,q\in M$ the function $f\: \TIL_p\to \RR$ defined by
\[f(v)=\tfrac12\cdot(\dist_q^2\circ\exp_p(v)-|v|^2)\] 
has convex suplevel sets.
\end{thm}