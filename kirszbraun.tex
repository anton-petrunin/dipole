\section{Cost-convex manifolds}\label{sec:cost-convex}

In this section we review necessary material we learned from \cite{MTW+CTIL}; see also the references there in.

\parbf{Cost-convex functions.}
Let $M$ be a Riemannian manifold.
Consider the \emph{cost function} $\cost\:M\times M\to \RR$ defined by \[\cost(x,y)=\tfrac12\cdot|x-y|_M^2.\]

A function $f\:M\to (-\infty,\infty]$ is called \emph{cost-convex} if 
for any $p\in M$ such that $f(p)<\infty$ there is $q\in M$ such that 
\[f(x)+\cost(q,x)\ge f(p)+\cost(q,p)\]
for any $x\in M$.

If $M$ has nonnegative curvature then any cost-convex function $f$ is $(-1)$-convex;
that is, $f''\ge-1$, which means that the inequality 
\[(f\circ\gamma)''\ge -1
\eqlbl{eq:f''=<-1}\]
holds in the barrier sense for any unit speed geodesic $\gamma$.
On the other hand, the inequality \ref{eq:f''=<-1} does not imply that $f$ is cost-convex.


\parbf{Subgradient.}
Let $f\: M\to (-\infty,\infty]$ be a semiconvex function defined on Riemannian manifold $M$.
Assume $f(p)$ is finite.
In this case the differential 
\[d_pf\:\T_p\zz\to (-\infty,\infty]\] 
is defined;
it is a convex positive homogeneous function defined on the tangent space $\T_p$.

A tangent vector $v\in \T_pM$ is a \emph{subgradient} of $f$ at $p$, briefly $v\in\ushort\nabla_pf$ if
\[\langle v,w\rangle\le d_pf(w)\]
for any $w\in \T_p$.
Note that the set $\ushort\nabla_pf$ is a convex subset of $\T_p$.

The subset of tangent vectors $v\in\T_p$ such that there is a minimizing geodesic $[p,q]$ in the direction of $v$ with length $|v|$ will be denoted as $\overline{\TIL}_p$. 
For $p,q$ and $v$ as above, we write $q\zz=\exp_pv$.

\begin{thm}{Definition}
A Riemannian manifold will be called \emph{cost-convex} if 
for any point $p$ and any cost-convex function $f$ which is finite at $p$ we have $\ushort\nabla_pf\subset\overline{\TIL}_p$
and for $q=\exp_pv$ the inequality 
\[\cost(q,p)-\cost(q,x)\ge f(x)-f(p)\]
holds for any $x\in M$.
\end{thm}

Note that any cost-convex manifold is nonnegatively curved.
According to \cite{FRV-Nec+Suf}, if $M$ is TCP then it is cost-convex.
The converse is unknown, likely it holds,
but there is a slightly stronger version of this condition which implies TCP.

\parbf{CTIL and MTW.}
Let us formulate other two conditions on Riemannian manifolds which are together equivalent to the cost-convexity.
These two conditions will be used in the proof.

Let $M$ be a Riemannian manifold.
The \emph{tangent injectivity locus} at the point $p\in M$ (briefly $\TIL_p$) is defined as the maximal open subset in the tangent space $\T_p$ such that for any $v\in\TIL_p$ the geodesic path $\gamma(t)=\exp_p(v\cdot t)$, $t\in [0,1]$ is minimizing.
If the tangent injectivity locus at any point $p\in M$ is convex we say that $M$ satisfies \emph{convexity of  tangent injectivity locus} or briefly $M$ is CTIL.

Note that $\overline{\TIL}_p$ defiend above is closure of $\TIL_p$ in $\T_p$.
Therefore, $M$ is CTIL if and only if the set $\overline{\TIL}_p$ is convex for any point $p\in M$.

The second condition is called MTW for Ma--Trudinger--Wang.
We will use its reformulation close to the one given by Cédric Villani \cite[2.6]{MTW+CTIL}; it can be proved the same way.

Assume $u,v\in \T_p$ and $w=\tfrac12\cdot(u+v)$
and $x=\exp_p u$, $y=\exp_pv$ and $q=\exp_pw$.
If the three geodesic paths $[p,x]$, $[p,y]$ and $[p,q]$ described by the paths 
$t\mapsto\exp_p(t\cdot u)$,  $t\mapsto\exp_p(t\cdot v)$, $t\mapsto\exp_p(t\cdot w)$ for $t\in[0,1]$ are minimizing, then $[p,q]$ is called \emph{median} of the hinge $[p\,^x_y]$.
Note that in a CTIL Riemannian manifold, any hinge has a median.

\begin{thm}{MTW condition}\label{MTW}
Assume $M$ be a CTIL Riemannian manifold. 
Then $M$ is MTW if and only if for a median $[p,q]$ of any hinge $[p\,^x_y]$ one of the following inequalities
\[
\left[
\begin{aligned}
|p-q|^2_M-|z-q|^2_M&\le |p-x|^2_M-|z-x|^2_M,
\\
|p-q|^2_M-|z-q|^2_M&\le |p-y|^2_M-|z-y|^2_M.
\end{aligned}
\right.
\]
holds for any $z\in M$.
\end{thm}

The following statement was proved by Gr\'egoire Loeper in \cite{loeper}, see also \cite[Proposition 2.5]{MTW+CTIL}.

\begin{thm}{Proposition}
A Riemannian manifold is cost-convex if and only if it is CTIL and MTW. 
\end{thm}

\section{On Kirszbraun's rigidity}
In the proof we will use the rigidity case of the generalized Kirszbraun theorem proved by Urs Lang and Viktor Schroeder in \cite{LS}, see also \cite{AKP}.

\begin{thm}{Kirszbraun rigidity theorem}\label{thm:kirszbraun-rigid}
Let $A$ be an Alexandrov space with nonnegative curvature.

Assume that for two point arrays $p,x_1,\dots,x_n\in A$ and $\~q, \~x_1,\dots,\~x_n\in \HH$ we have that 
\[|\~q-\~x_i|\ge |p-x_i|\]
for any $i$,
\[|\~x_i-\~x_j|\le |x_i-x_i|\]
for any pair $(i,j)$
and $\~q$ lies in the interior of the convex hull $\~K$ of $\~x_1,\dots,\~x_n$.

Then equalities hold in all the inequalities above.
Moreover there is an distance preserving map $f\:\~K\to A$ such that $f(\~x_i)=x_i$ and $f(\~q)=p$. 
\end{thm}

\parit{Proof.}
By the generalized Kirszbraun theorem, there is a short map $f\:A\to \HH$
such that $f(x_i)=\~x_i$.
Set  $\~p=f(p)$.
By assumptions
\[|\~q-\~x_i|\ge |\~p-\~x_i|.\]

Since $\~q$ lies in the interior of $K$, $\~q=\~p$.
It follows that the equality 
\[|\~q-\~x_i|= |p-x_i|.\]
holds for each $i$.

The remaining part will reduced to the case when $A$ is a Euclidean space;
which is left as an exercise.

According to Plaut's theorem (see \cite{AKP}) there is a dense G-delta set $S\subset A$ of points $s$ such that minimizing geodesics $[s,x_i]$ are uniquely defined and their directions at $s$ lie in a Euclidean subspace $E_s$ of the tangent cone $\T_sA$.

Fix $s\in S$ and set $v_i=\log_sx_i$.
Note that the gradient exponent 
\[\gexp_s\:E_s\to A\]
is a short map and $\gexp_s\:v_i\mapsto x_i$ for each $i$.
Note that $|v_i|_{E_s}=|s-x_i|_A$ for any $i$ therefore by triangle inequality
\[|v_i|_{E_s}\ge |p-x_i|_A-|p-s|_A.\]

Passing to the limit map as $s\to p$ we get (1) a collection of points in a Euclidean space $v_1,\dots,v_n\in E$ such that $|v_i|_E=|x_i-p|_A$
and (2) a short map $g\:E\to A$ such that $g\:v_i\to x_i$ and $g(0)=p$.

The composition $f\circ g\:E\to \HH$ is short
and by classical Kirszbraun rigidity it has to be distance preserving on the convex hull $\~K'$ of $v_i$.
Hence $\~K'$ is isometric to $\~K$ and the restriction $g|_{\~K'}$ is distance preserving. 
Hence the result.
\qeds
