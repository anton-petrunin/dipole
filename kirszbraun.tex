\section{Preliminaries}

\parbf{Cost-convex functions.}
Recall that complete length space with curvature bounded below in the sense of Alexandrov will be called \emph{Alexandrov space}.

Let $A$ be an Alexandrov space.
Consider the \emph{cost function} $\cost\:A\times A\to \RR$ defined by \[\cost(x,y)=\tfrac12\cdot|x-y|_A^2.\]
A function $f\:A\to \RR$ is called cost-convex if there is a subset of pairs $\mathcal{I}\subset A\times \RR$ such that
\[f(x)=\inf\set{r-\cost(p,x)}{(p,r)\in\mathcal{I}}.\]

If $A$ has nonnegative curvature then any cost-convex function $f$ is $(-1)$-convex, that is, it satisfies the inequality 
\[f''\ge -1\]
in the barrier sense (see \cite{AKP-book}).
On the other hand, the inequality $f''\ge -1$ does not imply that $f$ is cost-convex.


\parbf{Subgradient.}
Let $f\: A\to \bar\RR$ be a semiconvex function defined on Alexandrov space $A$
and $f(p)$ is finite.
Recall that differential $d_pf\:\T_p\to \bar\RR$ is defined.

A tangent vector $v\in \T_pA$ is called \emph{subgradient} of $f$ at $p$, briefly $v\in\ushort\nabla_pf$ if
\[\langle v,w\rangle\le d_pf(w)\]
for any $w\in \T_p$.
Note that the set $\ushort\nabla_pf$ is a convex subset of $\T_p$.

Recall that a tangent vector $v\in\T_p$ is called \emph{geodesic} if there is a minimizing geodesic $[p,q]$ in the direction of $v$ with length $|v|$;
in this case we write $q\zz=\exp_pv$.
The set of all geodesic vectors will be denoted by $\T'_p$;

An Alexandrov space will be called \emph{cost-convex} if 
for any cost-convex function $f$ any subgradient $v\in\ushort\nabla_pf$ is geodesic 
and for $q=\exp_pv$ the inequality 
\[\cost(q,p)-\cost(q,x)\ge f(x)-f(p)\]
holds for any $x\in A$.

\begin{thm}{Observation}
If $A$ is a cost-convex Alexandrov space, then for any $p\in A$ the set of geodesic tangent vectors $\T'_p$ is a convex subset of $\T_p$. 
\end{thm}

\parit{Proof.}
Fix $p\in A$ and consider the cost-concave function 
\[f=\inf\set{\cost(q,x)-\cost(q,p)}{q\in A}.\]
Note that any $\T'_p\subset \ushort\nabla_pf$.
Hence the statement follows.
\qeds

For a Riemannian manifold $M$,
the convexity of $\T'_p$ for any point $p\in M$ is the so called \emph{convexity of tangent injectivity locus}, briefly CTIL.
This property was considered in ??? as a necessary condition for the \emph{continuity of transport property}, briefly CTP.

The observation above implies that a cost-convex complete Riemannian manifold are CTIL.
Therefore, as it follows from \cite{loeper}, cost-convexity of complete Riemannian manifold is equivalent to MTW+CTIL condition.
Here MTW is stays for an other necessary condition for CTP introduced by Xi-Nan Ma, Neil Trudinger and Xu-Jia Wang, Xu-Jia in \cite{MTW}.

Likely cost-convexity is also sufficient for CTP.

\parbf{Kirszbraun rigidity theorem.}

\begin{thm}{Kirszbraun rigidity theorem}\label{thm:kirszbraun-rigid}
Let $A$ be a complete $\CBB[0]$ length space.

Assume that for two point arrays $p,x_1,\dots,x_n\in A$ and $\~q, \~x_1,\dots,\~x_n\in \HH$ we have that 
\[|\~q-\~x_i|\ge |p-x_i|\]
for any $i$,
\[|\~x_i-\~x_j|\le |x_i-x_i|\]
for any pair $(i,j)$
and $\~q$ lies in the interior of the convex hull $K$ of $\~x_1,\dots,\~x_n$.

Then equalities hold in all the inequalities above.
Moreover there is an distance preserving map $f\:K\to A$ such that $f(\~x_i)=x_i$ and $f(\~q)=p$. 
\end{thm}

\parit{Proof.}
By the generalized Kirszbraun theorem, there is a short map $f\:A\to \HH$
such that $f(x_i)=\~x_i$.
Set  $\~p=f(p)$.
By assumptions
\[|\~q-\~x_i|\ge |\~p-\~x_i|.\]

Since $\~q$ lies in the interior of $K$, $\~q=\~p$.
It follows that the equality 
\[|\~q-\~x_i|= |p-x_i|.\]
holds for each $i$.

According to ???, there is a short map $\HH\to \T_p$ such which admits a right inverse $\T_p\to \HH$ such that ???
\qeds
