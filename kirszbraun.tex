\section{Preliminaries}

\parbf{Proposed notation for trees}
A tree can be encoded using brackets say $(()()(()()))$ stays for the tree with 6 vertexes --- one for each pair of brakes where the vertexes connected by an edge if one of the pair goes directly in the other.
We will use shortcut
\[n=\underbrace{()\dots()}_{n\text{ times}},\]
so we can write $(2(2))$ for $(()()(()()))$.

If we need to label the vertexes of the tree we will do so using notation $(p,xy(q,vw))$.

\begin{thm}{Kirszbraun rigidity theorem}\label{thm:kirszbraun-rigid}
Let $A$ be a complete $\CBB[0]$ length space.

Assume that for two point arrays $p,x_1,\dots,x_n\in A$ and $\~q, \~x_1,\dots,\~x_n\in \HH$ we have that 
\[|\~q-\~x_i|\ge |p-x_i|\]
for any $i$,
\[|\~x_i-\~x_j|\le |x_i-x_i|\]
for any pair $(i,j)$
and $\~q$ lies in the interior of the convex hull $K$ of $\~x_1,\dots,\~x_n$.

Then equalities hold in all the inequalities above.
Moreover there is an distance preserving map $f\:K\to A$ such that $f(\~x_i)=x_i$ and $f(\~q)=p$. 
\end{thm}

\parit{Proof.}
By the generalized Kirszbraun theorem, there is a short map $f\:A\to \HH$
such that $f(x_i)=\~x_i$.
Set  $\~p=f(p)$.
By assumptions
\[|\~q-\~x_i|\ge |\~p-\~x_i|.\]

Since $\~q$ lies in the interior of $K$, $\~q=\~p$.
It follows that the equality 
\[|\~q-\~x_i|= |p-x_i|.\]
holds for each $i$.

According to ???, there is a short map $\HH\to \T_p$ such which admits a right inverse $\T_p\to \HH$ such that ...



\qeds