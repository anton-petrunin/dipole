\section{Kirszbraun's rigidity}

In the proof we will use the rigidity case of the generalized Kirszbraun theorem proved by Urs Lang and Viktor Schroeder in \cite{LS}, see also \cite{AKP}.

\begin{thm}{Kirszbraun rigidity theorem}\label{thm:kirszbraun-rigid}
Let $M$ be a complete Riemannian manifold with nonnegative sectional curvature.

Assume that for two point arrays $p,x_1,\dots,x_n\in M$ and $\~q, \~x_1,\dots,\~x_n\in \HH$ we have that 
\[|\~q-\~x_i|\ge |p-x_i|\]
for any $i$,
\[|\~x_i-\~x_j|\le |x_i-x_i|\]
for any pair $(i,j)$
and $\~q$ lies in the interior of the convex hull $\~K$ of $\~x_1,\dots,\~x_n$.

Then equalities hold in all the inequalities above.
Moreover there is an distance preserving map $f\:\~K\to M$ such that $f(\~x_i)=x_i$ and $f(\~q)=p$. 
\end{thm}

We reduce the theorem to the case $M=\RR^m$ which is left as an exercise.

\parit{Proof.}
By the generalized Kirszbraun theorem, there is a short map\footnote{Here and further \emph{short map} is a \emph{distance nonexpanding map}.} $f\:M\to \HH$
such that $f(x_i)=\~x_i$.
Set  $\~p=f(p)$.
By assumptions
\[|\~q-\~x_i|\ge |\~p-\~x_i|.\]

Since $\~q$ lies in the interior of $K$, we have that $\~q=\~p$ and the equality 
\[|\~q-\~x_i|= |p-x_i|.\]
holds for each $i$.

Set $v_i=\log_px_i$; that is, $t\mapsto \exp_p(t\cdot v_i)$ for $t\in[0,1]$ is a minimizing geodesic from $p$ to $x_i$ or, equivalently, $|v_i|=|p-x_i|$ and $\exp_pv_i=x_i$.
Recall that the gradient exponent $\gexp_p\:\T_p\to M$ is defined for any compete Riemannian manifold;
it is is a short map if $M$ has nonnegative curvature 
and $\gexp_p\:v_i\mapsto x_i$ for each $i$ (see \cite{AKP}).


The composition $f\circ \gexp_p\:\T_p\to \HH$ is short
and by classical Kirszbraun rigidity it has to be distance preserving on the convex hull $\~K'$ of $v_i$.
Hence $\~K'$ is isometric to $\~K$ and the restriction $g|_{\~K'}$ is distance preserving. 
Hence the result.
\qeds
