\section{On cost-convex manifolds}

In this section we review necessary material from ???.

\parbf{Cost-convex functions.}
Let $M$ be a Riemannian manifold.
Consider the \emph{cost function} $\cost\:M\times M\to \RR$ defined by \[\cost(x,y)=\tfrac12\cdot|x-y|_M^2.\]

A function $f\:M\to (-\infty,\infty]$ is called \emph{cost-convex} if there is a nonempty subset of pairs $\mathcal{I}\subset M\times \RR$ such that
\[f(x)=\sup\set{r-\cost(p,x)}{(p,r)\in\mathcal{I}}.\]

If $M$ has nonnegative curvature then any cost-convex function $f$ is $(-1)$-convex;
that is, $f''\ge-1$, which means that the inequality 
\[(f\circ\gamma)''\ge -1
\eqlbl{eq:f''=<-1}\]
holds in the barrier sense for any unit speed geodesic $\gamma$.
On the other hand, the inequality \ref{eq:f''=<-1} does not imply that $f$ is cost-convex.


\parbf{Subgradient.}
Let $f\: M\to (-\infty,\infty]$ be a semiconvex function defined on Riemannian manifold $M$.
Assume $f(p)$ is finite.
In this case the differential 
\[d_pf\:\T_p\zz\to (-\infty,\infty]\] 
is defined;
it is a convex positive homogeneous function defined on the tangent space $\T_p$.

A tangent vector $v\in \T_pM$ is a \emph{subgradient} of $f$ at $p$, briefly $v\in\ushort\nabla_pf$ if
\[\langle v,w\rangle\le d_pf(w)\]
for any $w\in \T_p$.
Note that the set $\ushort\nabla_pf$ is a convex subset of $\T_p$.

The subset of tangent vectors $v\in\T_p$ such that there is a minimizing geodesic $[p,q]$ in the direction of $v$ with length $|v|$ will be denoted as $\overline{\TIL}_p$. 
For $p,q$ and $v$ as above, we write $q\zz=\exp_pv$.

\begin{thm}{Definition}
A Riemannian manifold will be called \emph{cost-convex} if 
for any point $p$ and any cost-convex function $f$ which is finite at $p$ we have $\ushort\nabla_pf\subset\overline{\TIL}_p$
and for $q=\exp_pv$ the inequality 
\[\cost(q,p)-\cost(q,x)\ge f(x)-f(p)\]
holds for any $x\in M$.
\end{thm}

It is easy to see that any cost-convex maniflold is nonnegatively curved.
According to ???, if $M$ is CTP then it is cost-convex.
The converse is unknown, likely it holds,
but there is a slightly stronger version of this condition which implies CTP.

\parbf{CTIL and MTW.}
Let us formulate other two conditions on Riemannian manifolds which are together equivalent to the cost-convexity.
These two conditions will be used in the proof.

Let $M$ be a Riemannian manifold.
The \emph{tangent injectivity locus} at the point $p\in M$ (briefly $\TIL_p$) is defined as the maximal open subset in the tangent space $\T_p$ such that for any $v\in\TIL_p$ the geodesic path $\gamma(t)=\exp_p(v\cdot t)$, $t\in [0,1]$ is a minimizing.
If the tangent injectivity locus at any point $p\in M$ is convex we say that $M$ satisfies \emph{convexity of  tangent injectivity locus} or briefly $M$ is CTIL.

Note that $\overline{\TIL}_p$ defiend above is closure of $\TIL_p$ in $\T_p$.
Therefore, $M$ is CTIL if and only if the set $\overline{\TIL}_p$ is convex for any point $p\in M$.

The second condition is called MTW for Ma--Trudinger--Wang.
We will use its reformulation close to the one given by Cédric Villani \cite[2.6]{MTW+CTIL}; it can be proved the same way.

Assume $u,v\in \T_p$ and $w=\tfrac12\cdot(u+v)$
and $x=\exp_p u$, $y=\exp_pv$ and $q=\exp_pw$.
If the three geodesic paths $[p,x]$, $[p,y]$ and $[p,q]$ described by the paths 
$t\mapsto\exp_p(t\cdot u)$,  $t\mapsto\exp_p(t\cdot v)$, $t\mapsto\exp_p(t\cdot w)$ for $t\in[0,1]$ are minimizing, then $[p,q]$ is called \emph{median} of the hinge $[p\,^x_y]$.
Note that in a CTIL Riemannian manifold, any hinge has a median.

\begin{thm}{MTW condition}\label{MTW}
Assume $M$ be a CTIL Riemannian manifold. 
Then $M$ is MTW if and only if for a median $[p,q]$ of any hinge $[p\,^x_y]$ one of the following inequalities
\[
\left[
\begin{aligned}
|p-q|^2_M-|z-q|^2_M&\le |p-x|^2_M-|z-x|^2_M,
\\
|p-q|^2_M-|z-q|^2_M&\le |p-y|^2_M-|z-y|^2_M.
\end{aligned}
\right.
\]
holds for any $z\in M$.
\end{thm}

\section{On Kirszbraun's rigidity}
In the proof we will use the rigidity case of the generalized Kirszbraun theorem proved by Urs Lang and Viktor Schroeder in \cite{LS}, see also \cite{AKP}.

\begin{thm}{Kirszbraun rigidity theorem}\label{thm:kirszbraun-rigid}
Let $A$ be an Alexandrov space with nonnegative curvature.

Assume that for two point arrays $p,x_1,\dots,x_n\in A$ and $\~q, \~x_1,\dots,\~x_n\in \HH$ we have that 
\[|\~q-\~x_i|\ge |p-x_i|\]
for any $i$,
\[|\~x_i-\~x_j|\le |x_i-x_i|\]
for any pair $(i,j)$
and $\~q$ lies in the interior of the convex hull $\~K$ of $\~x_1,\dots,\~x_n$.

Then equalities hold in all the inequalities above.
Moreover there is an distance preserving map $f\:\~K\to A$ such that $f(\~x_i)=x_i$ and $f(\~q)=p$. 
\end{thm}

\parit{Proof.}
By the generalized Kirszbraun theorem, there is a short map $f\:A\to \HH$
such that $f(x_i)=\~x_i$.
Set  $\~p=f(p)$.
By assumptions
\[|\~q-\~x_i|\ge |\~p-\~x_i|.\]

Since $\~q$ lies in the interior of $K$, $\~q=\~p$.
It follows that the equality 
\[|\~q-\~x_i|= |p-x_i|.\]
holds for each $i$.

???

Consider the tangent vectors $v_i\in\T_p$ such that $\exp_pv_i=x_i$ for each $i$.
Note that these vectors are uniquely defined,
all  the vectors lie in an isometric copy of a Euclidean space
and 
\[|v_i-v_j|_{\T_p}=|x_i-x_j|_A.\]
In particular, the convex hull of $\{v_1,\dots,v_n\}$ in $\T_p$ is isometric to $\~K$,
so we can keep notation  $\~K$ for this convex hull.

Consider the gradient exponent $\gexp_p\:\T_p\to A$, see \cite{AKP-book}.
It is a short map, $\gexp_p0=p$ and $\gexp_p v_i=x_i$ for each $i$.
It remains to show that the restriction $\gexp_p|_{\~K}$ is distance-preserving.

Extend the sequence $v_1,\dots v_n$ to an infinite sequence of vectors $v_i\in\~K$ which is dense in $\~K$.
Set $x_i=\gexp_pv_i$ for each $i$.

Note that it is sufficient to show that the map $v_i\mapsto x_i$ is distance preserving.

From above,
\[|v_i-v_j|_{\T_p}=|x_i-x_j|_A\]
if $i,j\le n$; it provides a base for induction.
Assume 
\[|v_i-v_j|_{\T_p}=|x_i-x_j|_A\]
for all pairs $i,j\le k-1$.
Since the gradient exponent is short,
\[|v_i-v_k|_{\T_p}\ge|x_i-x_k|_A\]
for each $i\le k$.
From the first part of theorem we have 
\[|v_i-v_k|_{\T_p}=|x_i-x_k|_A.\]
It proves the induction step and hence the second statement follows.
\qeds
