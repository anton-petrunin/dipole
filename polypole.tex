\section{Polypolar comparison}\label{sec:all-tree}

Recall that a map $f\:W\to X$ between metric spaces is called \emph{submetry} if for any $w\in W$ and $r\ge 0$, we have 
\[f[B(w,r)_W]=B(f(w),r)_X,\]
where $B(w,r)_W$ denotes the ball with center $w$ and radius $r$ in the space $W$.
In other words submetry is a map which is 1-Lipschitz and 1-co-Lipschitz at the same time.
Note that any submetry is onto. 

\begin{thm}{Theorem}\label{thm:hilbert-quotient}
A separable metric space $X$ satisfies all tree comparison if and only if
$X$ is isometric to a target space of submetry defined of a subset  of the Hilbert space.
\end{thm}



\parit{Proof.}
The ``if'' part is left as an exercise;
let us prove the ``only if'' part.

Fix a point array $a_1,\dots, a_n$ in $X$.
Consider the complete graph $K_n$ with $\{1,\dots,n\}$ as the set of vertexes.


Let $\~ K_n\to K_n$ be the universal covering of the complete graph $K_n$.
Denote by $\~ V$ the set of vertexes of $\~ K_n$;
given a vetex $\~ v\in\~V$ denote by $v$ the corresponding vertex of $K_n$.

By multipolar comparison, we have the following:

\begin{enumerate}[$({*})$]
\item There is a map $f\:\~V\to\HH$ such that 
\[|f(\~v)-f(\~w)|_\HH\ge |a_v-a_w|_X\]
for any two vertexes $\~v,\~w\in \~ V$ and the equality holds if $(\~v,\~w)$ is an edge in $\~ K_n$.
\end{enumerate}

Since $X$ is separable, it contains a countable everywhere dense set $\{a_1,a_2,\dots\}$. 
Applying the statement above for $X_n=\{a_1,\dots a_n\}$, we get an isometric action $\Gamma_n\acts\HH$ and invariant sets $Y_n=f(\~V_n)\subset \HH$ such that $X_n$ is isometric to $Y_n/\Gamma_n$.

It remains to fix an ultra filter $\omega$ on $\NN$ and pass to the $\omega$-limit action on $\HH$. %???
\qeds

\begin{thm}{Proposition}
Suppose $G$ be a compct Lie group with bi-invariant metric, so the action $G\times G\acts G$ defined by $(h_1,h_2)\cdot g=h_1\cdot g h_2^{-1}$ is isometric. 
Then for any closed subgroup $H<G\times G$, the bi-quotient space $G/\!\!/H$ satisfies multipolar comparison.
\end{thm}

As a result we have many examples of spaces satisfying all tree comparison;
for example, since $\SS^n=\SO(n)/\SO(n-1)$, any round sphere satisfies multipolar comparison.

We present a proof suggested by Alexander Lytchak, it is simplified vesrion of the construction of Chuu-Lian Terng and Gudlaugur Thorbergsson given in \cite[Section 4]{terng-thorbergsson}.


\parit{Proof.}
Denote by $G^n$ the direct product of $n$ copies of $G$.
Consider the map $\phi_n\:G^n\to G$ defined by
\[\phi_n\:(\alpha_1,\dots,\alpha_n)\mapsto \alpha_1\cdots\alpha_n.\]
Note that $\phi_n$ is a quotient map for the $H\times G^{n-1}$-action on $G^n$ defined by
\[(\beta_0,\dots,\beta_n)\cdot(\alpha_1,\dots,\alpha_n)=(\gamma_1\cdot \alpha_1\cdot\beta_1^{-1},\beta_1\cdot\alpha_2\cdot\beta_2^{-1},\dots,\beta_{n-1}\cdot\alpha_n\cdot\beta_n^{-1}),\]
where $\beta_i\in G$ and $(\beta_0,\beta_n)\in H<G\times G$. 

Denote by $\rho_n$ the product metric on $G^n$ rescaled with factor $\sqrt{n}$.
Note that the quotient $(G^n,\rho_n)/(H\times G^{n-1})$ is isometric to $G/\!\!/H=(G,\rho_1)/\!\!/H$.

As $n\to\infty$ the curvature of $(G^n,\rho_n)$ converges to zero and its injectivity radius goes to infinity.
Therefore passing to the ultra-limit of $G^n$ as $n\to\infty$ we get the Hilbert space.
It remains to observe that the limit action has the required property.
\qeds